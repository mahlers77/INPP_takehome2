\documentclass[a4paper,11pt]{report}
\usepackage[margin=2.cm]{geometry}

\usepackage{amssymb}
\usepackage{amsmath}
\usepackage{amsfonts}
\usepackage{graphicx}

\usepackage{times}
%\usepackage{txfonts}
\usepackage{bm}

\usepackage{color}

\linespread{1}
\setlength{\parskip}{1.0ex}
\setlength{\parindent}{0.0ex}

\begin{document}

\begin{center}
{\large \textsc{Introduction to Nuclear and Particle Physics}} \\[0.5cm]
\Large{Take-Home Exam II\\
Due: Monday $20^{th}$ January 2020 at 13:00}
\end{center}

{\large \textsc{Problem 1 - Neutrino Oscillation in Matter} - 30 points}

In the lecture we studied neutrino oscillation in an (effective) two-flavour system. Two neutrino flavour states, $|\nu_e\rangle$ and $|\nu_\mu\rangle$, mix with two mass states, $|\nu_1\rangle$ and $|\nu_2\rangle$, such that
\[
\begin{pmatrix}|\nu_e\rangle\\|\nu_\mu\rangle\end{pmatrix} = \begin{pmatrix}\cos\theta&\sin\theta\\-\sin\theta&\cos\theta\end{pmatrix}\cdot\begin{pmatrix}|\nu_1\rangle\\|\nu_2\rangle\end{pmatrix}
\]
For a fixed momentum $p$, the Schroedinger equations of the two mass states $|\nu_i(t)\rangle$ (with $i=1$ or $i=2$) can be written as the matrix equation
\begin{equation}\label{eq:ex1a}
i\frac{\partial}{\partial t}\begin{pmatrix}|\nu_1(t)\rangle\\|\nu_2(t)\rangle\end{pmatrix} \simeq \begin{pmatrix}p + {m_1^2}/(2p)&0\\0&p + {m_2^2}/(2p)\end{pmatrix}\cdot\begin{pmatrix}|\nu_1(t)\rangle\\|\nu_2(t)\rangle\end{pmatrix}\,,
\end{equation}
where we used the relativistic approximation $E=\sqrt{p^2+m_i^2} \simeq p+m_i^2/(2p)$ in the last step. We use natural units with $c=1$ and $\hbar=1$. Similarly, the Schroedinger equations of the two flavour states $|\nu_\alpha(t)\rangle$ (with $\alpha=e$ or $\alpha=\mu$) can be written as a matrix equation
\begin{equation}\label{eq:ex1b}
i\frac{\partial}{\partial t}\begin{pmatrix}|\nu_e(t)\rangle\\|\nu_\mu(t)\rangle\end{pmatrix} = \begin{pmatrix}H_{ee}&H_{e\mu}\\H_{\mu e}&H_{\mu\mu}\end{pmatrix}\cdot\begin{pmatrix}|\nu_e(t)\rangle\\|\nu_\mu(t)\rangle\end{pmatrix}\,.
\end{equation}

1.1) Show, that the Hamiltonian matrix of vacuum oscillations can be written as
\begin{equation}\label{eq:ex1c}
{\bf H}_{\rm vac} \equiv\begin{pmatrix}H_{ee}&H_{e\mu}\\H_{\mu e}&H_{\mu\mu}\end{pmatrix} = a\begin{pmatrix}-\cos2\theta&\sin2\theta\\\sin2\theta&\cos2\theta\end{pmatrix} + b\begin{pmatrix}1&0\\0&1\end{pmatrix}\,,
\end{equation}
and determine the energy- and mass-dependent coefficients $a$ and $b$.

{\bf Hint:} Write the flavor states $|\nu_\alpha(t)\rangle$ as a superposition of mass states $|\nu_i(t)\rangle$ and use the evolution of Eq.~(\ref{eq:ex1a}) to derive the entries in Eq.~(\ref{eq:ex1b}).

1.2) Neutrino oscillations in matter are modified due to different interactions of neutrino in matter. In particular, the interactions of electron neutrinos and electron anti-neutrinos with electrons in matter can happen via charged current exchange. Draw the individual tree-level Feynman diagrams for the processes $\nu_e+e^- \to \nu_e +e^-$ and $\overline{\nu}_e+e^- \to \overline{\nu}_e +e^-$, that are not possible for $\nu_\mu$ or $\nu_\tau$ interactions with electrons. 

1.3) These matter effects introduce an additional potential term in the Hamiltonian:
\begin{equation}\label{eq:ex1d}
{\bf H}_{\rm mat} \equiv {\bf H}_{\rm vac} + \begin{pmatrix}V&0\\0&0\end{pmatrix}\,.
\end{equation}
Show that this sum can be re-written in the form similar to Eq.~(\ref{eq:ex1c})
\begin{equation}\label{eq:ex1e}
{\bf H}_{\rm mat} = \hat{a}\begin{pmatrix}-\cos2\hat\theta&\sin2\hat\theta\\\sin2\hat\theta&\cos2\hat\theta\end{pmatrix} + \hat{b}\begin{pmatrix}1&0\\0&1\end{pmatrix}\,,
\end{equation}
and give $\hat{a}$, $\hat{b}$ and $\sin2\hat{\theta}$ as a function of $a$, $b$, $\theta$ and $V_W$.

{\bf Hint:} Use the Eqs.~(\ref{eq:ex1c}), (\ref{eq:ex1d}) and (\ref{eq:ex1e}).

1.4) The neutrino flavour oscillation probability in matter looks very similar to that in vacuum. The oscillation amplitude is proportional to the new effective mixing angle as:
\begin{equation}
P_{\nu_\mu\to\nu_e} = \sin^22\hat{\theta}\sin^2\left(\frac{\hat{a}}{a}\frac{\Delta m^2\ell}{4p}\right)\,.
\end{equation}
Use the results in question 1.3 and show that the oscillation amplitude $\sin^22\hat{\theta}$ becomes maximal at a neutrino momentum corresponding to 
\begin{equation}
2a\cos2\theta = V\,.
\end{equation} 
This is the so-called Mikheyev-Smirnov-Wolfenstein (MSW) resonance.
\vspace{1cm}

{\large \textsc{Problem 2 - Helicity and Chirality} - 15 points}

A left-handed electron has a total energy $E$ and momentum ${\bf p}$ in the laboratory system. Its state $|u_\downarrow\rangle$ can be described as a superposition of left- and right-handed chirality states, $|u_\downarrow^\infty\rangle$ and $|u_\uparrow^\infty\rangle$, as 
\begin{equation}\label{eq:ex2a}
|u_\downarrow\rangle =\frac{1}{2}\sqrt{\frac{E+m}{E}}\left[ \left(1+\frac{|{\bf p}|}{E+m}\right)|u_\downarrow^\infty\rangle+\left(1-\frac{|{\bf p}|}{E+m}\right)|u_\uparrow^\infty\rangle\right]\,,
\end{equation}
where $m$ is the mass of the electron and we use natural units with $c=1$ and $\hbar=1$.

2.1) We want to consider the relativistic limit $m\ll E$. Show, that to leading order in $m/E$ the left-handed helicity state is given
\[
|u_\downarrow\rangle \simeq |u_\downarrow^\infty\rangle +\frac{m}{2E}|u_\uparrow^\infty\rangle
\]
%{\bf Hint:} Taylor-expand with respect to $m/E$.

2.2) Consider now the state of Eq.~(\ref{eq:ex2a}) in the rest-frame of the electron. What is the states composition in terms of left- and right-handed chirality states? What are the probabilities of finding the state in the left- or right-handed chirality state?

2.3) Show that in the case $m=0$, the left-handed helicity state becomes identical to the left-handed chirality state.
%2.2) Consider a Lorentz boost in the direction of the electron's momentum. The momentum of the electron in the new boosted reference system with observer velocity $v$ (measured in the original laboratory frame) is given in natural units as
%\begin{equation}
%p' = \gamma p - \gamma\beta E\,,
%\end{equation}  
%where $E$ is the electron's energy, $\beta = v/c$ and $\gamma = 1/\sqrt{1-\beta^2}$. Determine the critical velocity $v$ that is necessary to flip the helicity of the electron. If the electron had no mass, would it be possible to find an observer frame where the left-handed electron becomes a right-handed electron? Explain.

\vspace{1cm}

{\large \textsc{Problem 3 - Neutrino Scattering} - 15 points}

Consider the scattering processes $\nu_\mu +e^- \to \nu_\mu +e^-$. 

3.1) Draw the leading order tree-level diagram.

3.2) Calculate the ratio of the cross sections
\begin{equation}
\frac{\sigma(\nu_\mu +e_L^- \to \nu_\mu +e_L^-)}{\sigma(\nu_\mu +e_R^- \to \nu_\mu +e_R^-)}\,,
\end{equation} 
where the index $e^-_L$ is a left-handed and $e^-_R$ is a right-handed helicity electrons. (The neutrino is always assumed to be left-handed and we don't show the index there. Similarly, the anti-neutrino is always right-handed.)

{\bf Hint:} Start by considering the coupling of weak interactions to left-handed and right-handed leptons in the Standard Model. What are the relevant ``charges'' in the weak interactions for neutrinos and electrons?

3.3) Show that 
%\begin{equation}
%\frac{\sigma(\nu_\mu +e_L^- \to \nu_\mu +e_L^-)}{\sigma(\nu_\mu +\mu_L^- \to \nu_\mu +\mu_L^-)}\,.
%\end{equation} 
\begin{equation}
\frac{\sigma(\nu_\mu +e_L^- \to \nu_\mu +e_L^-)}{\sigma(\overline\nu_\mu +e_R^+ \to \overline\nu_\mu +e_R^+)}=1\,,
\end{equation}
to all orders in scattering theory.

{\bf Hint:} Consider the invariance of the Standard Model under certain combinations of discrete transformations.

\end{document}